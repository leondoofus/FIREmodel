\section{Nosedive}
Afin d'adapter et étendre le modèle pour simuler l'épisode \textit{Nosedive} de \texttt{BlackMirror}, nous pourrions rajouter des notes pour les consommateurs : n'importe quel agent (fournisseur ou consommateur) pourrait noter un consommateur selon l'attitude qu'il a eu lors d'une interaction donné (pas forcément à but lucratif). Après une interaction avec un fournisseur choisi grâce à la méthode recherche de témoins, le consommateur pourrait noter le témoin (en se basant sur la différence entre son témoignage et la performance reçue). Ces modifications nécessitent l'inclusion d'attitudes chez les agents. Il faudrait donc définir des groupes d'agents ayant un comportement défini. Similairement au type des fournisseurs, il y aurait de bons agents, des mauvais ou des imprévisibles. Cet attribut déterminerait leur caractère. 
Il faudrait ensuite restreindre certaines actions à un ensemble d'agents. C'est-à-dire que seulement les agents ayant une note supérieure à un certain seuil pourrait effectuer ces actions. 
Enfin, comme la notion de besoin n'existe pas chez les agents en Netlogo, nous devrions nous arranger pour que l'agent agisse stratégiquement pour obtenir ce dont il a besoin (i.e modifier temporairement son caractère dans le but de récolter de bonnes notes).  
