\section{Étude préliminaire}
\subsection{Question 1.1}
 La problématique est la suivante : Représenter la notion de confiance et de réputation dans la société. On veut un modèle décentralisé et robuste permettant de représenter les relations de confiances entre deux types d’agents (consommateur et fournisseur) et la réputation de chacun de ces agents au sein de la société ainsi qu’au niveau individuel. Il faudra pouvoir distinguer la fiabilité d’une information en fonction de sa source. On veut également que ce modèle puisse s’adapter efficacement face à un SMA ouvert, c’est à dire face à un environnement et des agents dynamiques. 

Les auteurs proposent, pour cela, un modèle modulaire séparant la source des informations en quatre grandes catégories : IT (interactions directs), CR (valeurs certifiées fournies par le fournisseur), WR (témoignages), et RT (notes fournies par des agents ayant un lien avec le fournisseur). Les agents consommateurs, pourront donc adapter leur choix en fonction de ces différentes données et de l'importance qu'ils leur accordent.
D'autre part, plusieurs caractéristiques ont été mis en place pour justifier le dynamisme du modèle : La présence d’un agent n’est pas fixe. Des entrées et des sorties aléatoires se feront au cours de la simulation. Enfin, la situation individuelle d’un agent peut aussi changer entre deux cycles. Pour cela, les agent peuvent se déplacer et les performances des fournisseurs peuvent changer (parfois radicalement).

\subsection{Question 1.2}
Le modèle combine ces 4 composantes pour diversifier la source des informations et de ce fait la rendre plus réaliste (ce qui rend le modèle sera plus robuste). Parallèlement, le modèle distingue ces composantes afin de leur appliquer un poids adapté (ce qui est plus représentatif également).
